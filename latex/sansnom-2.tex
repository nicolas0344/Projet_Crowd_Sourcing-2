%%% packages %%%
%%%%%%%%%%%%%%%%%%%%%%%%%%%%%%%%%%%%%%%%%%%%%%%%%%%%%%%%%%%%%%%%%%%%%%%%%%%%%%%  
\documentclass[frenchb]{report}
%\usepackage{natbib}
\usepackage[toc,page]{appendix}
\usepackage[dvipsnames]{xcolor}
\usepackage[french]{babel}
\usepackage{url}
\usepackage[utf8x]{inputenc}
\usepackage{graphicx}
\graphicspath{{images/}}
\usepackage{parskip}
\usepackage{fancyhdr}
\usepackage{fancyvrb}
\usepackage{vmargin}
\usepackage{xcolor}
\usepackage{bbm}
\usepackage{amsmath,amssymb}
\usepackage{amsthm}
\usepackage{dsfont}
\usepackage{stmaryrd}
\usepackage{systeme}
\usepackage{enumitem}
\usepackage{xcolor}
\usepackage{pifont}
\usepackage{textcomp}
\usepackage{calrsfs}
\usepackage[T1]{fontenc}
\usepackage[toc,page]{appendix}
\usepackage{lipsum}
\usepackage{verbatim}
\usepackage{listings}
\usepackage{adforn}
\usepackage{float}
\usepackage{subfig}
% Les packages necessaires pour faire le pseudo code
%%%%%%%%%%%%%%%%%%%%%%%%%
\usepackage{algorithm}
\usepackage{algorithmic}
%%%%%%%%%%%%%%%%%%%%%%%%%
% Je renomme les commandes en français
%%%%%%%%%%%%%%%%%%%%%%%%%%%%%%%%%%%%%%%%%%%%%%%%%%%%%%%%%%%%%%%
\renewcommand{\algorithmicrequire}{\textbf{Entrée(s) :}}
\renewcommand{\algorithmicreturn}{\textbf{retourner}}
\renewcommand{\algorithmicensure}{\textbf{Initialisation ;}}
\renewcommand{\algorithmicwhile}{\textbf{Tant que}}
\renewcommand{\algorithmicdo}{\textbf{Initialisation}}
\renewcommand{\algorithmicendwhile}{\textbf{fin du Tant que ;}}
\renewcommand{\algorithmicend}{\textbf{fin}}
\renewcommand{\algorithmicif}{\textbf{si}}
\renewcommand{\algorithmicendif}{\textbf{fin du si}}
\renewcommand{\algorithmicelse}{\textbf{sinon}}
\renewcommand{\algorithmicelsif}{\textbf{fin du sinon}}
\renewcommand{\algorithmicthen}{\textbf{alors}}
\renewcommand{\algorithmicthen}{\textbf{Étape E}}
\renewcommand{\algorithmicthen}{\textbf{Étape M}}
\renewcommand{\algorithmicfor}{\textbf{pour}}
\renewcommand{\algorithmicforall}{\textbf{pour tout}}
\renewcommand{\algorithmicto}{\textbf{à}}
\renewcommand{\algorithmicendfor}{\textbf{fin du pour}}
\renewcommand{\algorithmicdo}{\textbf{faire}}
\renewcommand{\algorithmicloop}{\textbf{boucler}}
\renewcommand{\algorithmicendloop}{\textbf{fin de la boucle}}
\renewcommand{\algorithmicrepeat}{\textbf{répéter}}
\renewcommand{\algorithmicuntil}{\textbf{jusqu’à}}
%%%%%%%%%%%%%%%%%%%%%%%%%%%%%%%%%%%%%%%%%%%%%%%%%%%%%%%%%%%%%%%
\setlength{\hoffset}{-18pt}        
\setlength{\oddsidemargin}{0pt} % Marge gauche sur pages impaires
\setlength{\evensidemargin}{9pt} % Marge gauche sur pages paires
\setlength{\marginparwidth}{54pt} % Largeur de note dans la marge
\setlength{\textwidth}{481pt} % Largeur de la zone de texte (17cm)
\setlength{\voffset}{-18pt} % Bon pour DOS
\setlength{\marginparsep}{7pt} % Séparation de la marge
\setlength{\topmargin}{0pt} % Pas de marge en haut
\setlength{\headheight}{13pt} % Haut de page
\setlength{\headsep}{10pt} % Entre le haut de page et le texte
\setlength{\footskip}{27pt} % Bas de page + séparation
\setlength{\textheight}{708pt} % Hauteur de la zone de texte (25cm)
\usepackage{hyperref}
\lstset{%
            inputencoding=utf8,
                extendedchars=true,
                literate=%
                {é}{{\'e}}{1}%
                {è}{{\`e}}{1}%
                {à}{{\`a}}{1}%
                {ç}{{\c{c}}}{1}%
                {œ}{{\oe}}{1}%
                {ù}{{\`u}}{1}%
                {É}{{\'E}}{1}%
                {È}{{\`E}}{1}%
                {À}{{\`A}}{1}%
                {Ç}{{\c{C}}}{1}%
                {Œ}{{\OE}}{1}%
                {Ê}{{\^E}}{1}%
                {ê}{{\^e}}{1}%
                {î}{{\^i}}{1}%
                {ô}{{\^o}}{1}%
                {û}{{\^u}}{1}%
                {ë}{{\¨{e}}}1
                {û}{{\^{u}}}1
                {â}{{\^{a}}}1
                {Â}{{\^{A}}}1
                {Î}{{\^{I}}}1
        }
    
    
\lstset{language=R,
  backgroundcolor=\color{lightgray},   % choose the background color; you must add \usepackage{color} or \usepackage{xcolor}; should come as last argument %MidnightBlue
   basicstyle=\small\ttfamily\color{white},        % the size of the fonts that are used for the code
  breakatwhitespace=false,         % sets if automatic breaks should only happen at whitespace
  breaklines=true,                 % sets automatic line breaking
  captionpos=b,                    % sets the caption-position to bottom
  commentstyle=\color{SpringGreen},    % comment style
  deletekeywords={data,frame,length,as,character},           % if you want to delete keywords from the given language
  extendedchars=true,              % lets you use non-ASCII characters; for 8-bits encodings only, does not work with UTF-8
  frame=single,	                   % adds a frame around the code
  keepspaces=true,                 % keeps spaces in text, useful for keeping indentation of code (possibly needs columns=flexible)
   keywordstyle=\color{Peach},       % keyword style
  morekeywords={kable,*,...},            % if you want to add more keywords to the set
  numbers=left,                    % where to put the line-numbers; possible values are (none, left, right)
  numbersep=5pt,                   % how far the line-numbers are from the code
  %numberstyle=\tiny\color{gray}, % the style that is used for the line-numbers
  rulecolor=\color{white},         % if not set, the frame-color may be changed on line-breaks within not-black text (e.g. comments (green here))
  showspaces=false,                % show spaces everywhere adding particular underscores; it overrides 'showstringspaces'
  showstringspaces=false,          % underline spaces within strings only
  showtabs=false,                  % show tabs within strings adding particular underscores
  stepnumber=2,                    % the step between two line-numbers. If it's 1, each line will be numbered
      % string literal style
  tabsize=2,	                   % sets default tabsize to 2 spaces
  title=\lstname,                  % show the filename of files included with \lstinputlisting; also try caption instead of title
  mathescape=true,
  escapechar=|
  }
%%%%%%%%%%%%%%%%%%%%%%%%%%%%%%%%%%%%%%%%%%%%%%%%%%%%%%%%%%%%%%%%%%%%%%%%%%%%%%%        


        
\makeatletter
\let\thetitle\@title
\let\theauthor\@author
\let\thedate\@date
\makeatother

%%% commandes mise en page %%%
%%%%%%%%%%%%%%%%%%%%%%%%%%%%%%%%%%%%%%%%%%%%%%%%%%%%%%%%%%%%%%%%%%%%%%%%%%%%%%%        
\newcommand{\ld}{\log_{2}}
\newcommand{\R}{\mathbbm{R}}
\newcommand{\N}{\mathbbm{N}}
\newcommand{\1}{\mathbbm{1}}
\newcommand{\E}{\mathbbm{E}}
\newcommand{\V}{\mathbbm{V}}
\newcommand{\prob}{\mathbbm{P}}
\newcommand{\Nc}{\mathcal{N}}
\newcommand{\Cc}{\mathcal{C}}
\newcommand{\K}{\mathcal{K}}
\newcommand{\Xti}{\widetilde{X_i}}
\newcommand{\Xtj}{\widetilde{X_j}}
\newcommand{\Xn}{\overline{X_n}}
\newcommand{\gn}{\hat{g_n}}
\newcommand{\n}{\mathcal{N}}
\newcommand{\lv}{\mathcal{L}}
\newcommand{\thetat}{\tilde{\theta}}
\newcommand{\indep}{\perp \!\!\! \perp}

\newcommand{\console}[1]{\colorbox{black}{\begin{minipage}[c]{1\linewidth}\textcolor{white}{\texttt{#1}}\end{minipage}}}

\newtheorem{prop}{Proposition}
\newtheorem{thm}{Théorème}
\newtheorem{cor}{Corollaire}
\newtheorem{lem}{Lemme}
\newtheorem{hyp}{Hypothèse}
\theoremstyle{definition}\newtheorem{defn}{Définition}
\theoremstyle{definition}\newtheorem{exm}{Exemple}
\theoremstyle{definition}\newtheorem{nota}{Notation}
\theoremstyle{definition}\newtheorem{rem}{Remarque}

\renewcommand{\qedsymbol}{\adfhangingflatleafright}


\begin{document}
%%% Pour l'annexe
\def\appendixpage{\vspace*{8cm}
\begin{center}
\Huge\textbf{Annexes}
\end{center}
}
\def\appendixname{Annexe}%
%%%

%%% Page de garde %%%
%%%%%%%%%%%%%%%%%%%%%%%%%%%%%%%%%%%%%%%%%%%%%%%%%%%%%%%%%%%%%%%%%%%%%%%%%%%%%%%%%%%%%%%%%
\begin{titlepage}
\begin{center}
\includegraphics[scale=0.6]{logo.png}
\hfill
\includegraphics[scale=0.35]{fds_logo.png}\\[3cm]
\linespread{1.2}\huge {\bfseries Projet Master 2 SSD }\\[0.5cm]
\linespread{1.2}\LARGE {\bfseries Crowd-Sourcing}\\[1.5cm]
\linespread{1}

{\large Rédigé par\\}
{\Large \textsc{pralon} Nicolas}\\
{\Large \textsc{thiriet} Aurelien}\\
{\large \emph{Encadrant :} Joseph \textsc{Salmon}}\\[1.5cm] 

\includegraphics[scale=0.7]{imag_logo.png}

\end{center}
\end{titlepage}
%%%%%%%%%%%%%%%%%%%%%%%%%%%%%%%%%%%%%%%%%%%%%%%%%%%%%%%%%%%%%%%%%%%%%%%%%%%%%%%%%%%%%%%%%
\tableofcontents
\newpage
%%%%%%%%%%%%%%%%%%%%%%%%%%%%%%%%%%%%%%%%%%%%%%%%%%%%%%%%%%%%%%%%%%%%%%%%%%%%%%%%%%%%%%%%%

\chapter*{Introduction}
\addcontentsline{toc}{part}{Introduction}

En science appliquée et notamment en statistique inférentielle, le recueil de données constitue une étape primordiale, il nous permet d'élaborer des modèles et de construire des estimateurs. Il convient à ce titre d'être attentif à l'observation des données.

Dans un cadre idéal, les modèles construit nécessitent un faible nombre de données dont l'observation peut être réalisée par des professionnels du domaine concerné. Toujours est-il que ces situations sont peu courantes et l'observation, par ces experts, d'un grand nombre de données n'est pas envisageable. C'est à ce titre que le "Crowd-Sourcing" est couramment utilisé, puisqu'un grand nombre d'intervenant permet de construire la base de donnée nécessaire. Toute fois des erreurs d'obervations quant à la nature des données peut-être commisent, il est alors essentiel de s'accorder sur une décision.

C'est à ce problème, que ce projet propose de présenter différentes solutions. Nous étudierons le modèle construit par DAWID et SKENE.


\newpage

\chapter{Modélisation du problème}

Dans ce chapitre, nous établissons le cadre nous permettant de traiter le problème de décision lorsque plusieurs obervations d'une même donnée sont fournit. 

Nous définissons alors l'exemple suivant : 

\begin{itemize}[label=\adfflowerleft]
	\item On dispose d'un esemble de patient $\left\{1,\cdots,I\right\}$ tous atteint d'une maladie, et on suppose le vecteur aléatoire $(X_i)_{i \in \{1,\cdots,I \}}$, tel que $\forall i$, $X_i$ à valeur dans $\left\{1,\cdots,J\right\}$ représente la maladie du partient $i$.
	\item On dispose egalement d'un esemble de médecin $\{1, \cdots, K\}$, et on considère $\forall i \in \{1, \cdots, I\}$ le vecteur aléatoire $(Y^k_i)_{k \in \{1, \cdots, K\}}$ le diagnostique du médecin $k$ au patient $i$, $Y^k_i$ à valeur dans $\{0,1\}^J$. 
	\item Chaque médecin répond indépendament des autres et la maladie de chaque patient est indépendante des autres. On note l'independance 2 à 2, $(Y^k_i)_k \indep$ et $(X_i)_i \indep$
\end{itemize}

Afin de diagnostiquer le meilleur traitement à chaque patient, on doit leur diagnostiquer une maladie. Toute fois nous pouvons nous retrouver dans la situation où le diagnostique différent d'un médecin à un autre, pour diverses raisons; erreur de mesure, erreur d'annotation, fatigue. Il convient dans ce cas de trouver une solution à la question : quelle maladie est atteint chaque patient ? \\

Dans le but de simplifier l'approche, nous considérons dans un premier temps le cas d'un medecin et d'un patient, nous pouvons également émettre l'hypothèse vraisemblable suivante : \\

\begin{hyp}
	Nous supposons que, $\forall j,k \in \llbracket 1,J \rrbracket \times \llbracket 1,K \rrbracket$, $Y^k_i | X_i = \alpha_i \sim Multinomiale((\pi^k_{\alpha_i,j})_{j \in \llbracket 1,J \rrbracket},1)$
\end{hyp}

Sachant que le partient $i$ est atteint de la maladie $\alpha_i$, la réponse du médecin $k$ suit une loi multinomiale. Le médecin peut, ou non, commettre une érreur de mesure même en connaissant la vraie maladie du patient. \\

Les paramètres $(\pi^k_{\alpha_i,j})_{j \in \llbracket 1,J \rrbracket}$ correspondent à la probabilité que le médecin $k$ diagnostique la maladie $j$ au patient $i$ sachant qu'il est atteint de la vraie maladie $\alpha_i$. On suppose que chaque médecin ne voit qu'une seule fois chaque patient.\\

On ainsi
\begin{center}
$\prob_{Y^k_i|X_i = \alpha_i} = \displaystyle \sum_{(n^k_{i,1}, \cdots, n^k_{i,J})} \underbrace{\frac{\big(\displaystyle \sum_{j=1}^J n^k_{i,j}\big)!}{\displaystyle \prod_{j=1}^J (n^k_{i,j})!}\displaystyle \prod_{j=1}^J \big(\pi^k_{\alpha_i,j}\big)^{n^k_{i,j}}}_{\textit{densité suivant} \displaystyle \sum \delta_{(n^k_{i,1}, \cdots, n^k_{i,J})}} \delta_{(n^k_{i,1}, \cdots, n^k_{i,J})}$
\end{center}

Avec $n^k_{i,j}$ le nombre de fois que le médecin $k$ à diagnostiqué la maladie $j$ au patient $i$\\

On a alors que la vraisemblance de $Y^k_i | X_i = \alpha_i$ est équivalent à $\displaystyle \prod_{j=1}^J \big(\pi^k_{\alpha_i,j}\big)^{n^k_{i,j}}$\\


Dans le but de se ramener au cas de plusieurs patients et plusieurs médecins, on considère ici un patient $i$ et tous les médecins $(1, \cdots, K)$.



\end{document}
